% !Mode:: "TeX:UTF-8"

\chapter{输入饱和约束下的预设阻尼比滑模反步控制器设计与分析}

本章在第2章“预设阻尼比滑模反步基准方案”的基础上,进一步面向工程系统中普遍存在的执行器输入幅值饱和问题,构建一种抗饱和的预设阻尼比滑模反步控制器,并给出可证明的稳定性与可操作的整定规则。

在真实平台中(例如伺服系统、无刷直流电机、板球平衡系统的舵机驱动等),控制输入通常满足幅值限制 $|u|\le u_{\max}$。若忽略饱和,控制律在瞬态阶段可能给出过大的开关项或补偿项,从而导致:
性能破坏:饱和引入输入失配,误差不再遵循预设的二阶目标动力学,超调/振荡可能显著变差;;稳定性风险:反步结构中含积分项或滤波环节时易产生“风up”,引发慢恢复或持续振荡;;滑模实现困难:开关项受限后到达条件被削弱,可能出现“到不了滑模面”或边界层内漂移。

因此,本章的核心贡献是:将输入饱和作为控制设计的一等公民,把“预设阻尼比”的解释性优势保留下来,同时让闭环在饱和存在时仍能证明一致最终有界,并给出输入不越界的整定准则与可复现实验对比指标。此外,为增强“滑模专属性”,本章进一步引入积分滑模思想,消除到达相,从而在初始瞬态阶段降低对大幅开关增益的依赖,使“抗饱和 + 滑模”形成逻辑闭环。

\vspace{0.5em}
\noindent 本章结构:
第\ref{sec:ch5_problem}节建立饱和模型与问题表述;
第\ref{sec:ch5_design}节给出抗饱和控制器结构(含抗饱和补偿与边界层);
第\ref{sec:ch5_stability}节给出稳定性与性能界分析;
第\ref{sec:ch5_ism}节给出积分滑模消除到达相的改进设计;
第\ref{sec:ch5_tuning_impl}节总结整定规则与离散实现;
第\ref{sec:ch5_sim}节给出可复现仿真方案与对比指标;
最后第\ref{sec:ch5_summary}节小结。

% ===================================================================
\section{问题描述:带输入饱和的不确定系统与控制目标}\label{sec:ch5_problem}

\subsection{带幅值饱和的二阶不确定系统}\label{sec:ch5_sat_sys}

延续第3章基准模型,考虑二阶不确定非线性系统
\begin{equation}\label{eq:ch5_sys}
\left\{
\begin{aligned}
  \dot{x}_1 &= x_2,\\
  \dot{x}_2 &= b\,u + f + d,\\
  y &= x_1,
\end{aligned}
\right.
\end{equation}
其中 $b\neq 0$ 已知或符号已知且有下界,$f$ 为未知/未建模项,$d$ 为外扰。

定义参考轨迹 $y_d$ 及误差
\begin{equation}\label{eq:ch5_e_def}
  e_1 = y-y_d,\qquad e_2 = \dot y-\dot y_d = x_2-\dot y_d.
\end{equation}

本章关注幅值饱和(最常见且最破坏瞬态的约束)。原因在于幅值限制对峰值/超调与到达过程影响最大;速率饱和通常可由执行器带宽与滤波近似建模,与本文主线耦合会显著增加推导复杂度,故暂不展开。令控制器计算得到的“期望输入”为 $u_c$,实际进入执行器的输入为
\begin{equation}\label{eq:ch5_sat_map}
  u = \mathrm{sat}(u_c)
  \triangleq
  \begin{cases}
    u_{\max}\,\mathrm{sgn}(u_c), & |u_c|>u_{\max},\\
    u_c, & |u_c|\le u_{\max},
  \end{cases}
\end{equation}
其中 $u_{\max}>0$ 为已知幅值上界。

为了把饱和效应显式纳入分析,定义饱和引起的输入失配(等效扰动)
\begin{equation}\label{eq:ch5_du_def}
  \Delta_u \triangleq u-u_c = \mathrm{sat}(u_c)-u_c.
\end{equation}
显然,当 $|u_c|\le u_{\max}$ 时 $\Delta_u=0$;当发生饱和时,$\Delta_u$ 与 $u_c$ 同号相反并使输入“被削弱”。该项会进入系统为匹配扰动 $b\Delta_u$。

\subsection{不确定性界与基本假设}\label{sec:ch5_assump}

与第3章一致,设存在名义模型 $\hat f$,并定义模型误差 $\tilde f\triangleq f-\hat f$ 与总不确定性
\begin{equation}\label{eq:ch5_delta}
  \Delta \triangleq \tilde f + d,\qquad |\Delta|\le \bar\Delta.
\end{equation}
同时假设参考信号 $y_d,\dot y_d,\ddot y_d$ 有界;$|b|\ge \underline b>0$。

饱和引入的 $\Delta_u$ 一般无法先验给出固定上界,因为它依赖 $u_c$;因此本章采用“两层策略”:
层1(不饱和区间保证):给出可操作的参数规则,使得在典型工况下 $|u_c|\le u_{\max}$,从而完全避免饱和;;层2(饱和时仍稳定):即使暂时发生饱和,也通过抗饱和补偿环节把 $\Delta_u$ 的影响吸收进 Lyapunov 分析,保证系统一致最终有界并可估计误差界。

\subsection{控制目标}\label{sec:ch5_goal}

本章目标可表述为:
稳定性:闭环所有信号有界,$e_1,e_2$ 一致最终有界(理想不饱和时可获得更强收敛性质);;预设阻尼比:在不饱和或仅小范围饱和时,$e_1$ 的主导瞬态近似满足;输入约束:尽可能保证 $|u|\le u_{\max}$ 且避免长时间饱和;必要时给出饱和下性能退化的可解释界。

% ===================================================================
\section{抗饱和预设阻尼比滑模反步控制器设计}\label{sec:ch5_design}

\subsection{滑模变量与二阶目标结构复用}

沿用第3章的“含积分终端”结构,定义积分状态
\begin{equation}\label{eq:ch5_q}
  q \triangleq \int_0^t e_1\,d\tau,\qquad \dot q = e_1,
\end{equation}
并定义滑模变量
\begin{equation}\label{eq:ch5_s}
  s \triangleq \dot e_1 + m_2 e_1 + m_1 q = e_2 + m_2 e_1 + m_1 q,
\end{equation}
其中 $m_1>0,m_2>0$ 用于嵌入阻尼比目标。注意恒等式
\begin{equation}\label{eq:ch5_sdot_key}
  \dot s = \ddot e_1 + m_2 \dot e_1 + m_1 e_1,
\end{equation}
因此只要能让 $\dot s$ 满足到达律或其扰动版本,$e_1$ 便呈现二阶可解释响应。

由系统\eqref{eq:ch5_sys}与误差定义\eqref{eq:ch5_e_def}可得
\begin{equation}\label{eq:ch5_sdot_expand}
\begin{aligned}
  \dot s
  &= \dot e_2 + m_2 \dot e_1 + m_1 e_1\\
  &= b\,u + f+d - \ddot y_d + m_2 e_2 + m_1 e_1\\
  &= b\,u + \hat f - \ddot y_d + m_2 e_2 + m_1 e_1 + \Delta.
\end{aligned}
\end{equation}

\subsection{控制律总体结构:名义项 + 滑模鲁棒项 + 抗饱和补偿}

本章将“计算得到的期望输入”写为
\begin{equation}\label{eq:ch5_uc_struct}
  u_c = u_0 + u_{\mathrm{sm}} + u_{\mathrm{aw}},
\end{equation}
其中:
$u_0$:名义补偿项,用于实现二阶目标结构;;$u_{\mathrm{sm}}$:滑模鲁棒项(带边界层),用于覆盖 $\Delta$;;$u_{\mathrm{aw}}$:抗饱和补偿项,专门吸收 $\Delta_u$ 对 $s$ 动力学的破坏。

实际进入系统的输入为 $u=\mathrm{sat}$。代回\eqref{eq:ch5_sdot_expand}后会出现 $b\Delta_u$,这是本章必须“显式处理”的关键。

\paragraph{(1)名义项}
令
\begin{equation}\label{eq:ch5_u0}
  u_0 = \frac{1}{b}\Big(-\hat f + \ddot y_d - m_2 e_2 - m_1 e_1\Big).
\end{equation}
若忽略不确定性与饱和,则 $u=u_0$ 时由\eqref{eq:ch5_sdot_expand}得到 $\dot s=\Delta\approx 0$,即误差呈现二阶目标结构。

\paragraph{(2)滑模鲁棒项(边界层实现)}
采用饱和型边界层以抑制抖振:
\begin{equation}\label{eq:ch5_sat_fun}
  \mathrm{sat}\!\left(\frac{s}{\phi}\right)=
  \begin{cases}
    \mathrm{sgn}(s), & |s|>\phi,\\
    \frac{s}{\phi}, & |s|\le \phi,
  \end{cases}
  \qquad \phi>0.
\end{equation}
设计
\begin{equation}\label{eq:ch5_usm}
  u_{\mathrm{sm}} = -\frac{1}{b}\Big(\rho+\varepsilon\Big)\,
  \mathrm{sat}\!\left(\frac{s}{\phi}\right),
\end{equation}
其中 $\rho\ge \bar\Delta$ 为鲁棒增益,$\varepsilon>0$ 为裕度。

\paragraph{(3)抗饱和补偿项:辅助系统法(最稳的可证写法)}
定义饱和失配信号
\begin{equation}\label{eq:ch5_delta_u_signal}
  \Delta_u = u-u_c.
\end{equation}
引入辅助动态(一个一阶稳定滤波/吸收器)
\begin{equation}\label{eq:ch5_xi_dyn}
  \dot \xi = -\lambda \xi + b\,\Delta_u,\qquad \lambda>0,
\end{equation}
并设计抗饱和补偿
\begin{equation}\label{eq:ch5_uaw}
  u_{\mathrm{aw}} = -\frac{1}{b}\,k_{\mathrm{aw}}\,\xi,\qquad k_{\mathrm{aw}}>0.
\end{equation}
直观理解:当发生饱和时,$b\Delta_u$ 会把 $s$ 的动力学“推偏”;我们用 $\xi$ 去跟踪并吸收这部分偏差,并把它以连续方式反馈回 $u_c$,从而避免积分项/滑模项在饱和时持续累积。

\paragraph{(4)最终控制律}
综合\eqref{eq:ch5_uc_struct}--\eqref{eq:ch5_uaw},得到
\begin{equation}\label{eq:ch5_final}
\boxed{
\begin{aligned}
  u_c &= \frac{1}{b}\Big(-\hat f + \ddot y_d - m_2 e_2 - m_1 e_1\Big)
        -\frac{1}{b}\Big(\rho+\varepsilon\Big)\mathrm{sat}\!\left(\frac{s}{\phi}\right)
        -\frac{1}{b}k_{\mathrm{aw}}\xi,\\
  u &= \mathrm{sat}(u_c),\\
  \dot \xi &= -\lambda \xi + b\,\Delta_u,\\
  s &= e_2+m_2 e_1+m_1 q,\qquad \dot q=e_1.
\end{aligned}}
\end{equation}

\subsection{闭环滑模变量动力学的“分解形式”}

将\eqref{eq:ch5_final}代入\eqref{eq:ch5_sdot_expand},得到
\begin{equation}\label{eq:ch5_sdot_closed}
  \dot s
  = -\Big(\rho+\varepsilon\Big)\mathrm{sat}\!\left(\frac{s}{\phi}\right)
    -k_{\mathrm{aw}}\xi + \Delta + b\,\Delta_u.
\end{equation}
再利用辅助系统\eqref{eq:ch5_xi_dyn},可把 $b\Delta_u$ 与 $\xi$ 捆绑进 Lyapunov 分析,这正是“抗饱和可证”的关键所在。

% ===================================================================
\section{稳定性与性能分析:饱和存在下的一致最终有界}\label{sec:ch5_stability}

本节给出两类结论:
结论A(饱和也不炸):即使发生饱和,闭环依然一致最终有界,并可给出误差界与参数的关系;;结论B(尽量不饱和):给出一组可操作的不饱和条件/整定规则,使得在典型工况下 $|u_c|\le u_{\max}$,从而恢复第2章的“预设阻尼比”理想解释。

\subsection{饱和下 Lyapunov 分析:$s$ 子系统}

取候选 Lyapunov 函数
\begin{equation}\label{eq:ch5_V}
  V \triangleq \frac12 s^2 + \frac{1}{2\kappa}\xi^2,\qquad \kappa>0.
\end{equation}
由\eqref{eq:ch5_sdot_closed}与\eqref{eq:ch5_xi_dyn}得
\begin{equation}\label{eq:ch5_Vdot_expand}
\begin{aligned}
  \dot V
  &= s\dot s + \frac{1}{\kappa}\xi\dot\xi\\
  &= s\Big[-\Big(\rho+\varepsilon\Big)\mathrm{sat}\!\left(\frac{s}{\phi}\right)
      -k_{\mathrm{aw}}\xi+\Delta+b\Delta_u\Big]
    +\frac{1}{\kappa}\xi\Big(-\lambda\xi+b\Delta_u\Big).
\end{aligned}
\end{equation}

注意到 $\mathrm{sat}$ 满足 $s\,\mathrm{sat}=|s|$或 $s^2/\phi$,因此总能写成
\begin{equation}\label{eq:ch5_sat_pos}
  s\,\mathrm{sat}\!\left(\frac{s}{\phi}\right)\ \ge\ \min\Big\{|s|,\frac{s^2}{\phi}\Big\}.
\end{equation}

此外,利用不确定性界 $|\Delta|\le \bar\Delta$,并选取 $\rho\ge \bar\Delta$,可得
\begin{equation}\label{eq:ch5_sDelta_bound}
  s\Delta \le |s|\,|\Delta|\le |s|\,\rho.
\end{equation}

把\eqref{eq:ch5_sDelta_bound}代入\eqref{eq:ch5_Vdot_expand},得到
\begin{equation}\label{eq:ch5_Vdot_step1}
\begin{aligned}
  \dot V
  &\le -\Big(\rho+\varepsilon\Big)s\,\mathrm{sat}\!\left(\frac{s}{\phi}\right) + |s|\,\rho
      -k_{\mathrm{aw}}s\xi + s\,b\Delta_u
      -\frac{\lambda}{\kappa}\xi^2 + \frac{b}{\kappa}\xi\Delta_u\\
  &\le -\varepsilon\,\min\Big\{|s|,\frac{s^2}{\phi}\Big\}
      -k_{\mathrm{aw}}s\xi
      -\frac{\lambda}{\kappa}\xi^2
      + b\Delta_u\Big|s+\frac{1}{\kappa}\xi\Big|.
\end{aligned}
\end{equation}

关键在最后一项:它只在饱和发生时非零。这里给出一种“保守但可写、可证”的处理方式:把 $b\Delta_u$ 视为有界信号,并通过辅助系统与参数选择把其影响压到最终界里。

由于 $u=\mathrm{sat}$,有 $|u|\le u_{\max}$;同时 $u_c$ 由控制器计算,其瞬时值可在线获得。于是 $|\Delta_u|=|u-u_c|\le |u|+|u_c|\le u_{\max}+|u_c|$。这虽然不是固定常数,但可用于推导“最终界与饱和程度相关”的结论。工程上更关心的是:当饱和仅在短暂瞬态出现,$\Delta_u$ 的能量有限,从而误差仍能被拉回。

为了把证明写得更干净,本章采用常见做法:将饱和等效项的影响“吸收进 $\xi$”,并用 Young 不等式把交叉项分解。对任意 $\eta_1,\eta_2>0$,有
\begin{equation}\label{eq:ch5_young}
  b|\Delta_u|\Big|s+\frac{1}{\kappa}\xi\Big|
  \le \frac{\eta_1}{2}s^2 + \frac{1}{2\eta_1}b^2\Delta_u^2
     +\frac{\eta_2}{2\kappa^2}\xi^2 + \frac{1}{2\eta_2}b^2\Delta_u^2.
\end{equation}
将其代入\eqref{eq:ch5_Vdot_step1},并选取 $k_{\mathrm{aw}}$ 使 $-k_{\mathrm{aw}}s\xi$ 可用 $s^2+\xi^2$ 上界,即可得到
\begin{equation}\label{eq:ch5_Vdot_uub}
  \dot V
  \le -c_1 s^2 - c_2 \xi^2 + c_3\,\Delta_u^2,
\end{equation}
其中 $c_1,c_2>0$ 可通过 $\lambda,k_{\mathrm{aw}},\kappa,\eta_1,\eta_2$ 选取保证,$c_3>0$ 为常数。

由\eqref{eq:ch5_Vdot_uub}可得典型 UUB 结论:当 $\Delta_u$ 的幅值或能量受限,则 $s$ 有界,且 $s$ 可被压到一个与饱和程度相关的邻域内。

\paragraph{解释:这意味着什么?}
若控制器整定得好,使 $|u_c|\le u_{\max}$,则\eqref{eq:ch5_Vdot_uub}退化为严格负定,$s$ 快速进入边界层并保持,此时恢复第2章的“预设阻尼比”解释;;若确实会饱和,则 $s$ 的最终界与 $\Delta_u^2$ 成正相关:饱和越狠,性能退化越明显;但由于有 $\xi$ 吸收,系统不会出现典型的 windup 失稳或长时间大幅振荡。

\subsection{$e_1$ 子系统有界性与预设阻尼比的“近似保持”}

由 $s=e_2+m_2 e_1+m_1 q$ 与 $e_2=\dot e_1$,得到
\begin{equation}\label{eq:ch5_e1q}
  \dot e_1 = -m_2 e_1 - m_1 q + s,\qquad \dot q=e_1.
\end{equation}
写成矩阵形式
\begin{equation}\label{eq:ch5_lin}
\begin{bmatrix}\dot e_1\\ \dot q\end{bmatrix}
=
\underbrace{\begin{bmatrix}
 -m_2 & -m_1\\
  1   &  0
\end{bmatrix}}_{A}
\begin{bmatrix}e_1\\ q\end{bmatrix}
+
\begin{bmatrix}1\\0\end{bmatrix}s.
\end{equation}
当 $m_1>0,m_2>0$ 时,$A$ 为 Hurwitz,因此 $s$ 有界 $\Rightarrow $ 一致最终有界。

进一步,由\eqref{eq:ch5_sdot_key}与\eqref{eq:ch5_sdot_closed}可把误差动力学写成
\begin{equation}\label{eq:ch5_e1_forced}
  \ddot e_1 + m_2 \dot e_1 + m_1 e_1
  = -\Big(\rho+\varepsilon\Big)\mathrm{sat}\!\left(\frac{s}{\phi}\right)
    -k_{\mathrm{aw}}\xi + \Delta + b\Delta_u.
\end{equation}
当不饱和或弱饱和时,右端可被压到较小范围,因此 $e_1$ 的主导动态仍近似呈现二阶欠阻尼响应,其阻尼比由
\[
\zeta=\frac{m_2}{2\sqrt{m_1}}
\]
解释;而强饱和时,右端增大,表现为“二阶目标动力学受到较强外力扰动”,超调与相位滞后会退化,这个退化程度可通过 $\Delta_u$ 的统计量(如峰值/均方)量化。

\subsection{不饱和区间保证:一组可操作的输入不越界整定准则}\label{sec:ch5_nosat_rule}

工程上常见诉求是:尽量不饱和,否则再强的鲁棒项也会被执行器“剪掉”。下面给出一组“可写进论文的”保守准则,用于在典型工况下保证 $|u_c|\le u_{\max}$。

由\eqref{eq:ch5_final},有
\begin{equation}\label{eq:ch5_uc_bound0}
  |u_c|
  \le \frac{1}{|b|}\Big(
  |\hat f| + |\ddot y_d| + m_2|e_2| + m_1|e_1|
  +\big|\mathrm{sat}\big|
  +k_{\mathrm{aw}}|\xi|
  \Big).
\end{equation}
注意 $\big|\mathrm{sat}\big|\le 1$。因此,只要能为典型工况给出信号上界
\[
|\hat f|\le \bar f_0,\quad |\ddot y_d|\le \bar y_2,\quad |e_1|\le \bar e_1,\quad |e_2|\le \bar e_2,\quad |\xi|\le \bar\xi,
\]
则一个充分条件为
\begin{equation}\label{eq:ch5_uc_suff}
  \frac{1}{\underline b}\Big(\bar f_0+\bar y_2+m_2\bar e_2+m_1\bar e_1+k_{\mathrm{aw}}\bar\xi\Big)\ \le\ u_{\max}.
\end{equation}

\paragraph{如何用这条式子?}
它告诉你:$m_1,m_2$ 不能无限大。你想要更快就会推高 $m_1\bar e_1$,更大阻尼也会推高 $m_2\bar e_2$;;它也告诉你:滑模鲁棒增益 $\rho$ 必须考虑输入余量。$\rho$ 选太大,输入必然更容易触顶;;抗饱和项 $k_{\mathrm{aw}}\bar\xi$ 是“保险丝”:它允许你在偶发饱和时不炸,但它本身也会占用输入预算,所以要配合 $\lambda$共同整定。

这就是本章“可证明 + 可整定”的核心:你可以把\eqref{eq:ch5_uc_suff}当作工程整定的上层约束,把“预设阻尼比”从纯理想指标推进到“考虑执行器极限的可落地指标”。

% ===================================================================
\section{积分滑模改进:消除到达相以减小饱和风险}\label{sec:ch5_ism}

你在讨论里提到“让创新点更滑模”,最典型的滑模专属强点之一就是积分滑模:通过构造积分型滑模面,使系统从 $t=0$ 起就位于滑模面上,从而:
初始瞬态不需要很大的开关增益去“拉到滑模面”,自然更不容易触发饱和;;抗不确定性从一开始就生效(这在强扰动、强初值偏差时尤其关键)。

\subsection{积分滑模面构造}

考虑将第\ref{sec:ch5_design}节的 $\dot s$ 展开式\eqref{eq:ch5_sdot_expand}分解为“名义部分 + 不确定部分”。定义名义控制 $u_0$ 仍为\eqref{eq:ch5_u0},并构造一个积分状态 $w$:
\begin{equation}\label{eq:ch5_w_def}
  \dot w
  \triangleq
  b\,u_0 + \hat f - \ddot y_d + m_2 e_2 + m_1 e_1,
  \qquad w(0) = s(0).
\end{equation}
再定义积分滑模变量
\begin{equation}\label{eq:ch5_sism}
  s_I \triangleq s-w.
\end{equation}
由初值 $w(0)=s(0)$ 可得
\[
s_I(0)=0,
\]
即系统从初始时刻起就满足积分滑模条件(无到达相)。

对\eqref{eq:ch5_sism}求导并结合\eqref{eq:ch5_sdot_expand}与\eqref{eq:ch5_w_def},得到一个极其关键的简化:
\begin{equation}\label{eq:ch5_sI_dyn}
  \dot s_I
  = b\big(u_{\mathrm{sm},I}+u_{\mathrm{aw}}+\Delta_u\big) + \Delta.
\end{equation}
也就是说,积分滑模变量的动态中名义部分被完全消掉,只剩下我们需要用滑模项处理的不确定性与(可能存在的)饱和失配。

\subsection{ISM 下的滑模项与抗饱和补偿}

令控制结构改为
\begin{equation}\label{eq:ch5_ism_control}
  u_c = u_0 + u_{\mathrm{sm},I} + u_{\mathrm{aw}},\qquad u=\mathrm{sat},
\end{equation}
其中
\begin{equation}\label{eq:ch5_usmI}
  u_{\mathrm{sm},I}
  = -\frac{1}{b}\Big(\rho+\varepsilon\Big)\mathrm{sat}\!\left(\frac{s_I}{\phi}\right),
\end{equation}
抗饱和环节仍采用\eqref{eq:ch5_xi_dyn}--\eqref{eq:ch5_uaw},只需将 $s$ 替换为 $s_I$ 即可。

此时由\eqref{eq:ch5_sI_dyn}可得
\begin{equation}\label{eq:ch5_sI_closed}
  \dot s_I
  = -\Big(\rho+\varepsilon\Big)\mathrm{sat}\!\left(\frac{s_I}{\phi}\right)
    -k_{\mathrm{aw}}\xi + \Delta + b\Delta_u.
\end{equation}
这与\eqref{eq:ch5_sdot_closed}形式一致,但重要差异在于:$s_I(0)=0$,因此不会出现传统滑模那种“先把 $s$ 拉到 0”的到达过程——这在工程上通常意味着:初始时刻所需开关能量显著降低,更抗饱和。

\subsection{ISM 改进带来的工程含义}

若你把第2章/第4章的实验对比设计成“同样的 $y_d$”,通常会观察到:ISM 版本的控制输入峰值更小、触顶次数更少;;由于饱和触发更少,误差更贴近二阶目标结构,阻尼比预设的可解释性更稳;;写论文时你可以把这一节作为“滑模专属强点”:在输入受限场景下,用 ISM 消除到达相以降低输入峰值,从而把鲁棒滑模与抗饱和耦合起来。

% ===================================================================
\section{参数整定与离散实现要点}\label{sec:ch5_tuning_impl}

本节给出“能直接拿去调参/写实验”的规则,目标是:让你在第5章做实验时不靠玄学。

\subsection{阻尼比指标到 $(m_1,m_2)$ 的选择}

与第3章一致:
\begin{equation}\label{eq:ch5_m_from_zeta}
  m_1=\omega_{n,d}^2,\qquad m_2=2\zeta_d\omega_{n,d},
\end{equation}
其中 $\zeta_d\in(0,1)$ 为预设阻尼比,$\omega_{n,d}$ 决定响应速度。

抗饱和提示:执行器越弱,$\omega_{n,d}$ 的上限越低;否则 $u_0$ 项中 $m_1 e_1,m_2 e_2$ 会把 $u_c$ 推到饱和。

\subsection{鲁棒增益 $\rho$ 与边界层 $\phi$}

先给 $\rho$ 一个够用但不夸张的初值,使其覆盖你估计的 $\bar\Delta$;;然后用输入预算约束\eqref{eq:ch5_uc_suff}回推:如果必饱和,就优先减小 $\rho$ 或降低 $\omega_{n,d}$;;边界层 $\phi$ 与噪声/采样强相关:$\phi$ 太小会放大开关抖动,间接引发饱和;工程上建议从“传感噪声标准差的 3--5 倍”起步,再微调。

\subsection{抗饱和环节 $\lambda,k_{\mathrm{aw}}$ 的选择}

$\lambda$ 控制 $\xi$ 的泄放速度,越大越“快忘掉饱和误差”,但太大可能把离散噪声放大成高频输入;;$k_{\mathrm{aw}}$ 控制补偿强度:太小则 windup 抑制不足,太大则 $u_{\mathrm{aw}}$ 本身占用输入预算。

一个可操作的经验起点:
\begin{equation}\label{eq:ch5_aw_rule}
  \lambda \in [5\omega_{n,d},\,20\omega_{n,d}],\qquad
  k_{\mathrm{aw}} \in [0.5,\,5].
\end{equation}
然后在实验中观察两类曲线:饱和触发次数与恢复时间。若频繁触顶且恢复慢,增大 $k_{\mathrm{aw}}$ 或 $\lambda$;若输入高频抖动明显,减小 $\lambda$ 或增大 $\phi$。

\subsection{离散实现建议}

对控制器\eqref{eq:ch5_final},离散实现可用前向欧拉:
\begin{equation}\label{eq:ch5_discrete}
\begin{aligned}
  q[k+1]   &= q[k] + T_s e_1[k],\\
  \xi[k+1] &= \xi[k] + T_s\big(-\lambda \xi[k] + b\,\Delta_u[k]\big),\\
  s[k]     &= e_2[k] + m_2 e_1[k] + m_1 q[k],\\
  u_c[k]   &= \frac{1}{b}\Big(-\hat f[k] + \ddot y_d[k] - m_2 e_2[k] - m_1 e_1[k]\Big)\\
            &\quad -\frac{1}{b}\Big(\rho+\varepsilon\Big)\mathrm{sat}\!\left(\frac{s[k]}{\phi}\right)
            -\frac{1}{b}k_{\mathrm{aw}}\xi[k],\\
  u[k]     &= \mathrm{sat}(u_c[k]).
\end{aligned}
\end{equation}

两个工程小坑(强烈建议写进论文):
$q$ 的积分漂移:建议对 $q$ 做限幅 $|q|\le q_{\max}$ 或加入泄漏项 $\dot q=e_1-\epsilon_q q$,否则长时间微小偏差会把 $q$ 推到很大,导致 $m_1 q$ 把输入顶到饱和;;$e_2$ 的获取:若速度不可测,不要直接差分 $\dot y$,应使用第3章的观测器/微分跟踪器得到平滑估计,否则噪声会通过 $m_2 e_2$ 与滑模项放大并触发饱和。

% ===================================================================
\section{仿真验证方案与对比指标(可直接搬到第5章实验设计)}\label{sec:ch5_sim}

为验证“抗饱和 + 预设阻尼比 + 滑模/ISM”的有效性,本节给出一套可复现对比方案。即使你把主要实验放到第5章,本节也能提供足够篇幅与逻辑闭环。

\subsection{对比控制器组}

建议至少比较四组(写论文最好看):
G1:第2章基准 PDR-SMB(无抗饱和),$u=\mathrm{sat}$ 直接硬饱和;;G2:本章 AS-PDR-SMB;;G3:本章 ISM-AS-PDR-SMB(积分滑模 + 抗饱和);;G4(可选):经典 anti-windup PI 或 DOB 方案(作为工程对照基线)。

\subsection{工况设置}

设置两类典型工况:轻饱和工况,选择较温和的参考 $y_d$,使得理想情况下 $|u_c|$ 大部分时间低于 $u_{\max}$,用于验证“阻尼比预设是否仍准确”;强饱和工况,人为提高初始误差或给较快的轨迹,使输入多次触顶,用于验证“抗饱和是否能抑制 windup 并更快恢复”。

\subsection{评价指标(建议写进表格)}

建议记录以下指标,论文表达清晰:
超调量:$M_p = \frac{\max_t |e_1|}{|e_1(0)|}$ 或针对阶跃响应定义;;调节时间:$t_s$;;饱和强度:饱和占空比 $D_{\mathrm{sat}}=\frac{1}{T}\int_0^T \mathbb{I}dt$;;输入峰值:$u_{\mathrm{pk}}=\max_t |u|$;;恢复性能:饱和解除后误差回到误差带的时间 $t_{\mathrm{rec}}$;;抖振度量:输入高频能量;;ISM 额外指标:初始阶段的 $u_c$ 峰值与 $D_{\mathrm{sat}}$,通常 ISM 会显著降低它们。

\subsection{结果组织建议}

建议用“表 + 图”结构(即使你现在还没图,也可以先留空位,后续补图):
一张表:四组控制器的关键参数与指标;;轻饱和工况:两张图;;强饱和工况:两张图。

这些内容后续完全可以“迁移”到第5章的真实实验里:把仿真里的 $u_{\max}$ 对应舵机 PWM 或电压限幅,把 $D_{\mathrm{sat}}$ 统计改成 PWM 触顶次数即可。

% ===================================================================
\section{本章小结}\label{sec:ch5_summary}

本章针对工程系统普遍存在的输入饱和约束,提出了抗饱和的预设阻尼比滑模反步控制器 AS-PDR-SMB。通过在第3章 PDR-SMB 基准结构上引入饱和映射 $u=\mathrm{sat}$,并构造辅助动态 $\dot\xi=-\lambda\xi+b$ 及其反馈补偿项 $u_{\mathrm{aw}}=-\frac{1}{b}k_{\mathrm{aw}}\xi$,实现对饱和引起的输入失配的吸收与抑制,从而在饱和存在时仍可证明闭环信号一致最终有界,并使误差动力学尽可能保持二阶目标结构的可解释性。

进一步地,为增强滑模方法的“专属性”并降低初始瞬态对大开关增益的依赖,本章引入积分滑模思想,通过构造积分状态 $w$ 并定义 $s_I=s-w$ 使得 $s_I(0)=0$,从而消除传统滑模的到达相,减少饱和触发风险并提升预设阻尼比规律在工程约束下的可保持性。最后,本章给出了面向工程实现的整定规则(含输入预算约束与抗饱和参数建议)、离散实现注意事项以及可复现实验/仿真对比指标,为第5章的板球平衡系统实验验证提供了直接可落地的方案。

% Local Variables:
% TeX-master: "../thesis"
% TeX-engine: xetex
% End:
