% !Mode:: "TeX:UTF-8"

\chapter{绪论}[Introduction]

\section{研究背景与意义}[Background and motivation]

复杂工程对象对高性能控制具有很高的需求。复杂工程对象(如航天器姿轨控制、机电伺服系统、平台稳定与精密运动控制等)普遍具有非线性、强耦合、不确定性与外部扰动显著等特点。在此类系统中,控制器不仅要保证闭环稳定与鲁棒性,还必须满足工程上对瞬态过程的明确指标约束,例如:超调量限制、收敛(响应)速度、稳态误差界、抗扰恢复时间以及控制输入约束等。尤其在精密控制场景(高精度跟踪、稳定平台、视觉伺服、柔性机构等)中,瞬态性能的波动会直接导致任务失败、机构冲击与能耗上升,因此“可解释、可预设、可验证”的性能设计方法具有突出的工程价值,这一需求构成了后续讨论的现实出发点。

传统线性控制理论(极点配置、频域整形、线性二次调节与线性二次高斯等)在系统近似线性且参数确定时能够提供较成熟的设计流程,但面对强非线性、不确定性以及状态不可测的情况,单纯依赖线性化往往难以兼顾全局性能与鲁棒性。在此背景下,面向非线性系统的系统化设计方法(如反步控制、滑模控制、自适应控制、扰动观测器控制、预设性能控制等)成为研究热点。
\cite{ref38,ref39,ref40,ref41}


反步控制具有一系列优势与局限。反步控制是一类针对严格反馈结构非线性系统的经典递推设计方法。其核心思想是:逐层引入虚拟控制量并构造误差变量,通过构造逐层扩展的李雅普诺夫函数来证明闭环系统稳定性,从而得到控制律。\cite{ref39,ref40,ref1,ref2}反步法的突出优点在于:其一,设计过程递推、可解释,结构清晰且可系统化推广,便于在高阶系统中形成通用设计框架;其二,可在李雅普诺夫框架下给出一致最终有界或渐近稳定等性质,稳定性证明友好;其三,便于与自适应估计、鲁棒补偿、观测器、滤波器等机制融合,形成组合设计,这些特点使其成为工程应用中的重要基石。然而,反步控制在工程实现中也存在典型困难:当系统阶次较高或包含未知非线性项时,虚拟控制的逐层求导会引发“爆炸性复杂性”,导致控制器表达式过于复杂、对噪声敏感、实现代价高\cite{ref5,ref6,ref7};另一方面,经典反步稳定性分析主要关注误差收敛与有界性,通常难以直接把“超调量/阻尼比/响应速度”等瞬态指标显式纳入参数化设计中,工程上往往仍需要依赖经验调参来获得满意的瞬态性能。

以阻尼比为核心的瞬态指标预设思想。对二阶系统而言,阻尼比与超调量、衰减速度之间存在明确的解析关系。在经典欠阻尼情形下,误差可近似为标准二阶形式
\begin{equation}\label{eq:intro_second_order}
  \ddot e + 2 \zeta \omega_n \dot e + \omega_n^2 e = 0,
\end{equation}
其中 $\zeta \in (0,1)$ 为阻尼比、$\omega_n>0$ 为自然频率,则超调量 $M_p$ 满足
\begin{equation}\label{eq:intro_mp}
  M_p = \exp\!\left(-\frac{\pi \zeta}{\sqrt{1-\zeta^2}}\right).
\end{equation}
该关系表明:在欠阻尼范围内,阻尼比越大通常超调越小,系统响应更“平稳”;阻尼比越小则响应更“快”但振荡更强。工程设计中若能将阻尼比(或等价的超调约束)作为显式指标嵌入控制律与稳定性分析,则可将“凭经验调参”转化为“按指标选参”,显著提升控制器的可用性、可迁移性与工程交付效率。\cite{ref38}

但对非线性不确定系统而言,误差很难天然呈现为 \eqref{eq:intro_second_order} 的标准形式;同时鲁棒补偿项、观测误差、扰动注入等都会破坏理想二阶结构。因此,需要一种能够在非线性鲁棒控制框架下依然保持“阻尼比—瞬态性能”可解释映射的理论工具。\cite{ref57,ref58,ref59,ref60,ref64,ref65,ref66}

本文研究问题与意义。本文面向一类二阶不确定非线性系统,围绕“预设阻尼比的可解释控制设计”开展研究。核心目标是在反步控制递推框架中引入二阶类李雅普诺夫稳定性准则,将阻尼比相关参数直接嵌入李雅普诺夫导数不等式,从而实现对超调量、响应速度与稳态精度的统一调节;进一步针对工程系统中普遍存在的状态不可测与外部扰动,引入滑模观测器与扰动观测器形成可落地的控制方案\cite{ref46,ref29,ref36,ref51},并通过板球平衡系统实验验证方法的有效性,这一目标为后续章节的技术路线提供了逻辑支点。

本文研究具有以下意义:理论层面提出二阶类李雅普诺夫设计思路,将阻尼比指标引入非线性鲁棒控制的稳定性分析与参数设计中,增强控制设计的可解释性;方法层面在反步结构中融合滑模鲁棒机制以及观测器/扰动补偿机制,构建适用于不确定非线性系统的预设阻尼比控制框架;工程层面提供一套“指标可预设、参数可解释、实现可落地”的控制器设计与验证流程,降低调参成本,提高在真实平台上的应用可行性。

\section{国内外研究现状}[Related work]

\subsection{反步控制与复杂性抑制方法}[Backstepping and complexity reduction]

反步控制在严格反馈系统中具有广泛应用。经典反步设计通常通过构造误差变量 $z_i$ 与虚拟控制 $\alpha_i$,并递推构造李雅普诺夫函数
\begin{equation}
  V_i = V_{i-1} + \frac{1}{2} z_i^2,
\end{equation}
使得 $\dot V_i$ 满足负定或负半定条件,从而保证稳定性。但在高阶系统中,$\alpha_i$ 的高阶求导会使表达式迅速膨胀,导致计算复杂且对噪声敏感。

为降低复杂度,研究者提出了多种改进方法:例如动态面控制通过一阶滤波器替代虚拟控制的显式求导;命令滤波反步通过滤波器与误差补偿项实现对虚拟控制导数的近似;微分跟踪器用于获得平滑导数估计,减轻噪声放大。上述方法的共同目标是在尽量不破坏稳定性证明结构的前提下,降低实现难度并提高噪声鲁棒性。\cite{ref1,ref2,ref3,ref4,ref5,ref6,ref7}

\subsection{滑模控制与滑模反步鲁棒结构}[Sliding mode and sliding-mode backstepping]

滑模控制以其对匹配不确定项与外部扰动的鲁棒性著称,常通过设计滑模面 $s(\cdot)$ 与到达律实现有限时间到达与不敏感性。\cite{ref44,ref45,ref46}将滑模机制与反步设计相结合,可形成滑模反步、终端滑模以及与反步耦合的相关结构,使得系统在不确定性与扰动存在时仍具备更强的误差收敛与抗扰能力。\cite{ref8,ref9,ref10,ref11,ref12,ref18}

然而,经典滑模项往往带来抖振问题\cite{ref48,ref49};同时滑模强鲁棒性与瞬态性能之间存在耦合:增大开关增益通常提升抗扰但会加剧抖振与输入冲击,反之则可能削弱鲁棒性。因此,在滑模反步设计中,如何在鲁棒性、输入平滑性与瞬态指标之间实现可解释的权衡,是工程应用中的关键问题。

\subsection{瞬态性能约束与预设性能控制}[Transient performance constraints and PPC]

预设性能控制通过性能函数与误差变换将误差约束在预设的时间变化包络内,实现对收敛速度、稳态误差与超调的间接约束。该类方法在理论上可实现“误差始终满足给定约束”,但在严格反馈系统、鲁棒补偿与观测误差并存的情况下,误差变换可能引入复杂的非线性项与奇异性处理问题,且参数选择的可解释性与工程易用性仍需增强。\cite{ref62,ref63}

与 PPC 不同,本文采用“阻尼比—超调”这一经典二阶系统指标映射,将阻尼比相关参数直接嵌入李雅普诺夫导数不等式,使得控制器参数选择可与超调指标建立直接、可解释的联系;并通过二阶类李雅普诺夫稳定性准则在非线性系统中形成可证明的设计规则。\cite{ref57,ref58,ref59,ref60,ref64,ref65,ref66}

进一步地,预设瞬态或漏斗控制在约束瞬态响应方面也形成了独立的研究脉络。\cite{ref69,ref71,ref73}

\subsection{观测器与扰动补偿技术}[Observers and disturbance compensation]

工程系统中常存在部分状态不可测(如速度、加速度、内模状态等),且测量存在噪声与采样周期限制,直接依赖全状态反馈难以实现。滑模观测器(包括高阶滑模结构)具有有限时间收敛与较强鲁棒性,适合用于估计不可测状态与导数信息\cite{ref29,ref30,ref31,ref46};同时,为避免求导噪声放大问题,可结合跟踪微分器与滤波机制获得平滑的导数估计。

另一方面,摩擦、负载变化、外界干扰与建模误差等可统一视为广义扰动。扰动观测器或扩张状态观测器通过在线估计未知扰动并在控制律中补偿,可显著提升抗扰性能与跟踪精度\cite{ref24,ref51,ref36,ref32,ref33,ref34,ref35,ref52,ref53,ref54,ref55,ref56}。在鲁棒反步框架中,若能将状态观测与扰动估计进行联合设计,则可同时应对“不可测 + 扰动”双重挑战。

\subsection{小结与研究空白}[Summary and research gap]

综上:反步控制提供了系统化稳定性设计框架,滑模机制提供了鲁棒补偿能力,观测器/扰动观测器提供了工程可实现性与抗扰增强手段;但在现有研究中,“将阻尼比/超调作为显式可调指标并给出可证明参数设计规则”,以及“在观测误差与扰动存在时仍保持指标可解释映射”仍具有研究空间。本文的研究工作旨在填补上述空白,并形成可在真实平台上验证的完整技术链条。

\section{研究对象、问题描述与课题来源}[System, problem description and project source]

\subsection{研究对象与控制目标}[System and control objectives]

本文研究对象为一类二阶不确定非线性系统。系统可包含未知非线性项、外部扰动以及部分状态不可测等工程因素。控制目标是设计控制输入 $u$,使系统输出 $y$ 对参考信号 $y_d$ 实现有界跟踪,并使闭环误差响应满足预设阻尼比对应的瞬态指标(尤其是超调量可调且可解释)。

在应用层面,本文选择板球平衡系统作为验证平台。该系统通过舵机控制平板在 $x$、$y$ 方向倾斜角度,从而间接调节小球在平面上的滚动位置与速度;系统具有明显的非线性、耦合与参数时变特性,同时视觉测量与离散采样带来噪声与延迟,是检验控制器工程可行性的典型对象。

\subsection{课题来源与研究阶段性成果}[Project source and staged outputs]

本课题围绕“非线性系统预设阻尼比滑模反步控制方法”展开,已形成阶段性研究成果,包括相关学术论文与专利申请,并完成实验平台搭建与部分实验验证,为后续深化研究与论文撰写奠定基础。

\section{研究内容与技术路线}[Research content and technical route]

\subsection{总体思路}[Overall idea]

本文以“指标可预设、参数可解释、实现可落地”为主线:首先建立适用于二阶不确定非线性系统的预设阻尼比设计准则;随后在反步框架中引入滑模鲁棒机制,并分别面向“全状态可测”“部分状态不可测”“外部扰动显著”三类情形构建控制方案;最后在板球平衡系统上开展点跟踪与轨迹跟踪实验,验证阻尼比预设对超调与响应速度的调节能力,以及观测器/扰动补偿对鲁棒性的提升。

\subsection{主要研究内容}[Main tasks]

本文围绕以下三方面展开研究:


;预设阻尼比滑模反步控制:
  在标准反步设计中引入二阶类李雅普诺夫稳定性判据,将阻尼比相关参数嵌入李雅普诺夫导数不等式;结合滑模面设计与鲁棒补偿项,得到可调瞬态指标(超调量/响应速度)的滑模反步控制律,并给出一致最终有界或渐近收敛分析。

;基于状态观测器的预设阻尼比滑模反步控制:
  针对部分状态不可测与求导依赖问题,构造滑模状态观测器估计不可测状态;同时引入微分跟踪器或滤波结构替代虚拟控制的解析求导,抑制噪声放大并降低工程实现复杂度;分析观测误差对闭环性能、超调调节与稳态精度的影响,给出参数选取与增益设计建议。

;基于扰动观测器与状态观测器的预设阻尼比滑模反步控制:
  面向外部扰动与建模误差,将未知项视为广义扰动并构造扰动观测器在线估计;在控制律中进行主动补偿以提升抗扰与恢复能力;进一步联合状态观测器实现不可测状态与扰动的同步估计,在离散采样与噪声条件下保持良好瞬态响应与跟踪精度。


\subsection{技术路线}[Technical route]

本文技术路线可概括为:

;模型与指标映射:建立系统模型与误差动力学;给出阻尼比与超调量的关系,并构造二阶类李雅普诺夫稳定性判据,使阻尼比参数进入稳定性分析框架。
;控制律递推设计:基于反步递推结构设计虚拟控制与实际控制输入;引入滑模面与鲁棒项,确保不确定性与扰动条件下的稳定性与误差有界性。
;观测与补偿扩展:针对状态不可测,设计滑模观测器并分析观测误差;针对扰动显著,设计扰动观测器并进行补偿,形成更强鲁棒的闭环结构。
;仿真与实验验证:在仿真平台中对参数规律、性能边界与鲁棒性进行预验证;在真实板球系统上完成固定目标点与时变轨迹(如圆形等)跟踪实验,统计超调量、响应时间、稳态误差与抗扰恢复特性,并对比不同控制方案。


\subsection{验证平台与实验方案概述}[Platform and experimental plan]

板球平衡系统通过嵌入式控制器实现实时控制与数据采集。实验设计覆盖两类典型场景:

;常值目标点跟踪:验证系统在不同阻尼比设定下的超调量、收敛速度与稳态误差;
;时变轨迹跟踪:验证动态目标下的跟踪精度、相位滞后与抗扰能力(例如圆形轨迹等)。

同时,针对多种控制器结构进行对比实验,在不同扰动条件下评估响应性能与鲁棒性,为方法有效性提供系统性证据。

\section{关键科学问题与研究难点}[Key challenges]

结合不确定非线性系统的控制需求与真实平台实现约束,本文研究主要面临以下难点:

\subsection{指标预设与非线性鲁棒性的统一}[Unifying preset specifications with nonlinear robustness]

在不确定非线性系统中,误差动力学难以保持标准二阶形式;鲁棒项与观测误差会改变瞬态过程。如何在保证稳定性与鲁棒性的同时,使阻尼比参数对超调与振荡程度仍保持清晰、可解释的调节作用,是本文的首要科学问题。

\subsection{参数设计的可操作性与工程可复现}[Practical parameter selection and reproducibility]

工程调试中常存在“高维参数搜索”问题:需要在响应速度、超调抑制、抗扰性与输入平滑性等多目标约束下寻找合适参数组合。若缺乏可解释的选参规则,控制器虽理论可行但工程落地困难。本文旨在通过二阶类李雅普诺夫判据给出明确的参数设计逻辑,并在实验中总结参数变化对性能的规律。

\subsection{离散采样、噪声与观测误差影响}[Sampling, noise and observer errors]

板球系统依赖视觉测量,采样频率有限且测量存在噪声,直接求导会显著放大噪声;观测器虽然可估计不可测状态,但观测误差与有限采样会影响滑模项与鲁棒补偿效果。如何在离散系统中实现稳定、平滑且有效的观测与补偿,是工程实现的关键挑战。

\section{本文主要创新点}[Contributions]

本文的主要创新点概括如下:


;提出一种面向预设阻尼比/超调量的二阶类李雅普诺夫设计思路,将瞬态性能指标以可调参数形式嵌入稳定性分析与控制器设计过程,实现“按指标选参”的可解释设计。
;在反步递推结构中引入滑模面与鲁棒补偿项,构建预设阻尼比滑模反步控制框架,提升对不确定非线性项、参数时变与外部扰动的抑制能力。
;针对部分状态不可测与高阶导数依赖问题,结合滑模状态观测器与导数估计机制,形成更易工程实现的观测器型预设阻尼比控制器,并给出观测误差影响分析。
;进一步联合扰动观测器与状态观测器,实现不可测状态与外部扰动的同步估计与主动补偿,提升闭环系统在扰动环境下的跟踪性能、超调抑制与恢复能力。
;基于板球平衡系统开展点跟踪与轨迹跟踪实验验证,形成从理论到实现的完整证据链,支撑方法的可行性与工程适用性。


\section{论文结构安排}[Organization]

全文结构安排如下:

第1章为绪论,阐述研究背景与意义,总结国内外相关研究进展,明确本文研究对象、研究内容、技术路线、关键难点与创新点,并给出全文结构安排。

第2章给出系统模型与问题描述,介绍严格反馈结构非线性系统的表述方式;建立预设阻尼比与超调量之间的指标映射关系,并提出二阶类李雅普诺夫稳定性判据,为后续控制器设计与稳定性证明提供统一理论基础。

第3章研究预设阻尼比滑模反步控制器的设计方法与稳定性分析,给出参数与瞬态性能指标之间的对应关系,并讨论鲁棒项对超调调节与输入平滑性的影响。

第4章在第3章基础上引入状态观测器与导数估计机制,构建观测器型预设阻尼比滑模反步控制器,分析观测误差对闭环稳定性、超调与稳态精度的影响,并给出相应设计建议。

第5章进一步考虑外部扰动与建模误差,研究扰动观测器与状态观测器联合的预设阻尼比滑模反步控制方法,给出扰动估计误差界下的稳定性与性能分析。

第6章开展板球平衡系统实验验证与结果讨论,覆盖固定目标点与时变轨迹跟踪两类典型任务,对比不同控制方案在不同阻尼比设定与扰动条件下的超调量、响应速度、稳态误差与鲁棒性表现。

最后给出全文结论与未来工作展望,总结本文主要研究成果与工程意义,并讨论后续可进一步研究的方向(例如更一般结构系统扩展、约束处理、输入饱和与离散实现改进等)。

\section{本章小结}[Summary]

本章介绍了不确定非线性系统高性能控制的工程背景,指出传统反步控制在复杂性与瞬态指标显式设计方面的局限,强调以阻尼比为核心的指标预设思想在工程应用中的价值。随后综述了反步控制、滑模鲁棒设计、预设性能控制以及观测器与扰动补偿技术的研究现状,明确本文的研究空白与目标。最后给出了本文的研究内容、技术路线、关键难点、创新点与论文结构安排,为后续章节展开理论推导与实验验证奠定基础。

% Local Variables:
% TeX-master: "../thesis"
% TeX-engine: xetex
% End:
