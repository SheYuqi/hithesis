% !Mode:: "TeX:UTF-8"
\hitsetup{
  statesecrets={公开},
  ctitlecover={非线性系统预设阻尼比滑模反步控制方法研究},
  ctitle={非线性系统预设阻尼比滑模反步控制方法研究},
  cxueke={工学},
  csubject={控制科学与工程},
  caffil={航天学院},
  cauthor={佘宇琪},
  csupervisor={郑晓龙},
  cstudentid={23S004010},
  cstudenttype={学术学位论文},
  cdate={2025年6月},
  etitle={Backstepping Control With Predefined Damping Ratios: A Second-Order Lyapunov Method},
  exueke={Engineering},
  esubject={Control Science and Engineering},
  eaffil={\emultiline[t]{School of Astronautics\\Harbin Institute of Technology}},
  eauthor={Yuqi She},
  esupervisor={Professor Xiaolong Zheng},
  edate={June, 2025},
  estudenttype={Master Thesis},
  ckeywords={预设阻尼比, 滑模反步控制, 二阶类李雅普诺夫方法, 状态观测器, 扰动观测器},
  ekeywords={Predefined damping ratio, Sliding-mode backstepping, Second-order Lyapunov method, State observer, Disturbance observer},
}

\begin{cabstract}
针对一类严格反馈结构的不确定非线性系统,围绕系统瞬态指标“超调量/阻尼比”难以在现代李雅普诺夫框架下直接设计的问题,本文研究预设阻尼比的滑模反步控制方法。首先引入具有参数选择规则的二阶类李雅普诺夫稳定性判据,将输出误差的收敛过程约束为满足二阶微分不等式,使阻尼比可由控制器参数近似预先指定,从而实现对超调量与响应速度的可调设计。在此基础上,构造包含积分项的滑模面并给出鲁棒反步控制律,在存在未建模动态与外部扰动时证明闭环信号实用稳定且输出具有期望阻尼比。进一步面向工程中“部分状态不可测”的情形,引入高阶Levant滑模观测器对不可测状态进行有限时间估计,并结合微分跟踪器替代虚拟控制函数的高阶求导以降低实现复杂度;同时考虑外扰影响,设计扰动观测器对广义扰动在线估计并进行前馈补偿,提升系统鲁棒性与跟踪精度。最后以板球平衡系统为实验对象,在STM32F407平台上实现控制算法并完成定点/轨迹跟踪验证,结果表明所提方法能够在不同阻尼比设定下实现可调超调与稳定快速跟踪。
\end{cabstract}

\begin{eabstract}
This thesis studies a sliding-mode backstepping control framework for strict-feedback nonlinear systems with uncertainties, with an emphasis on explicitly shaping transient performance via a predefined damping ratio. A second-order Lyapunov method is introduced to impose a second-order differential inequality on the output error dynamics, which provides practical stability while enabling the damping ratio  to be approximately predetermined through specific controller parameters. Based on this criterion, a robust backstepping controller with a properly constructed sliding surface is developed and analyzed in the presence of unmodeled dynamics and external disturbances. For practical scenarios with partially unmeasurable states, a high-order Levant sliding-mode observer is employed to estimate the unavailable states in finite time, and a differentiator/tracking mechanism is incorporated to avoid the “explosion of complexity” caused by repeated analytic differentiation of virtual control laws. Moreover, a disturbance observer is integrated to estimate and compensate generalized disturbances online, improving robustness and tracking accuracy. Experiments on a ball-and-plate platform implemented on an STM32F407 controller validate that the proposed approach achieves stable tracking with tunable overshoot and response speed under different predefined damping-ratio settings.
\end{eabstract}