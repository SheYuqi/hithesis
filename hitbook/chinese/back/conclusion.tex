% !Mode:: "TeX:UTF-8"

\begin{conclusions}

    本文面向一类二阶不确定非线性系统(并推广至严格反馈结构系统),围绕“瞬态指标可预设、参数选择可解释、工程实现可落地”这一主线,
    研究并构建了预设阻尼比滑模反步控制框架及其观测器与扰动补偿扩展方案。
    针对传统反步控制在不确定系统中难以直接将超调量等瞬态指标纳入可证明设计、以及高阶导数依赖导致实现复杂度迅速增加的问题,
    本文以二阶系统阻尼比与超调量的定量关系为桥梁,引入二阶类李雅普诺夫稳定性判据,将阻尼比(或等价超调量)作为显式设计目标嵌入闭环误差动力学,
    并通过滑模鲁棒项、状态观测与扰动估计补偿等技术,使得在不确定性、部分状态不可测以及外部扰动存在时仍能保证稳定性与性能可调性。
    在板球平衡系统平台上开展点跟踪与轨迹跟踪实验验证,系统评估了不同方案在不同预设阻尼比与扰动条件下的动态响应、超调量与稳态精度表现。
    
    \vspace{0.5\baselineskip}
    \noindent 本文的主要研究结果与创新性成果可概括如下(分条给出,兼顾定性与定量评价口径):
    
    
;提出了面向预设阻尼比/超调量的二阶类李雅普诺夫设计方法,建立了“指标--参数”的可解释映射规则。\\
      针对非线性系统中瞬态指标难以直接参数化设计的问题,本文从二阶标准型出发,
      给出了阻尼比 $\zeta$ 与超调量 $M_p$、峰值时间 $t_p$ 等指标的定量关系,
      并通过二阶类李雅普诺夫等式/不等式判据,将误差动力学约束为
      \[
        \ddot{e} + m_2 \dot{e} + m_1 e \approx 0,
      \]
      从而使 $(m_1,m_2)$ 与 $\zeta$(或 $M_p$)建立直接对应关系:
      \[
        \zeta = \frac{m_2}{2\sqrt{m_1}},\qquad
        M_p \approx \exp\!\left(-\frac{\pi \zeta}{\sqrt{1-\zeta^2}}\right).
      \]
      由此实现了“按指标选参”的设计范式,将传统依赖经验的调参过程转化为可解释、可复用的参数选取规则,
      为后续控制器设计与实验对比提供了统一的理论基准。
    
;建立了预设阻尼比滑模反步控制框架,实现不确定系统中“稳定性 + 瞬态可调”的统一设计。\\
      在反步递推结构中引入滑模鲁棒补偿,提出预设阻尼比滑模反步控制器,
      通过构造包含积分项的滑模变量
      \[
        s=\dot e_1+m_2 e_1+m_1\int_0^t e_1\,d\tau,
      \]
      使得滑模建立后误差主导动态满足目标二阶标准型,从而获得可预设阻尼比(或等价超调量/峰值时间)的瞬态响应。
      在不确定性与外部扰动存在时,利用到达律与边界层饱和函数实现鲁棒性与可实现性折中,
      并证明了滑模变量可达、闭环信号有界以及误差一致最终有界。
      该结果实现了在严格反馈不确定非线性系统中的“递推稳定化 + 二阶瞬态指标嵌入 + 鲁棒补偿”一体化设计。
    
;提出了观测器型预设阻尼比滑模反步控制方案,解决部分状态不可测与高阶导数依赖的工程实现问题。\\
      面向速度/高阶状态不可测、以及反步递推中虚拟控制导数难以解析求取导致的“爆炸性复杂性”,
      本文在预设阻尼比滑模反步框架下引入滑模状态观测器与微分跟踪器,
      构建观测器型控制器,实现仅输出可测条件下的闭环控制。
      通过将观测误差与跟踪误差统一纳入二阶类微分不等式形式,给出了观测误差对最终误差界与瞬态偏差的显式影响路径,
      并解释了在观测误差有界的条件下,预设阻尼比特性可在工程意义上近似保持。
      该方案显著降低了对全状态测量的依赖,提升了方法的工程适用性与可移植性。
    
;提出了扰动观测器与状态观测器联合的预设阻尼比滑模反步控制方案,实现广义扰动主动补偿并降低滑模增益需求。\\
      针对仅依赖滑模鲁棒项往往需要较大切换增益从而引起抖振与稳态精度受限的问题,
      本文将未知非线性与外部扰动统一为广义扰动,通过扩张状态观测器/扰动观测器在线估计并在控制律中主动补偿,
      构建 DO-PDR-SMB 控制结构,使滑模鲁棒项仅需覆盖扰动估计误差与离散实现误差等剩余小项。
      理论上证明了在扰动变化率有界等条件下,扰动估计误差一致最终有界,从而显著缩小误差最终界并改善预设阻尼比特性的保持程度;
      工程上为降低抖振、提高稳态精度与能效提供了可操作的选参依据(观测器带宽、边界层宽度与鲁棒增益的协同设计)。
    
;在板球平衡系统上完成了点跟踪与轨迹跟踪实验验证,系统评估了不同方案在不同预设阻尼比与扰动条件下的性能表现。\\
      以板球平衡系统作为验证平台,本文围绕固定目标点跟踪与封闭轨迹(圆形/三角形)跟踪任务开展对比实验,
      从超调量、响应速度(上升/峰值时间)、稳态误差与抗扰性能等角度评估 PDR-SMB、O-PDR-SMB 与 DO-PDR-SMB 三类方案。
      实验结果表明:在阻尼比设定变化时系统动态响应的振荡程度与峰值特性随参数变化呈现可预期趋势;
      在扰动或负载变化条件下,引入扰动观测补偿可有效降低鲁棒增益需求并提升稳态精度;
      在仅输出可测场景下,观测器型方案能够保持稳定并获得与全状态方案相近的工程可用性能。
      这些结果为本文方法的可实现性与工程应用潜力提供了支撑。
    
    
    \vspace{0.5\baselineskip}
    \noindent 尽管本文工作在理论与实验层面给出了较为完整的框架与验证,但仍存在进一步深入研究的空间。未来工作可从以下几个方向展开:
    
    
;更一般不确定性与多变量系统扩展: 将本文以匹配不确定性为主的分析框架进一步扩展到
      不匹配不确定性、强耦合多输入多输出系统与约束系统(输入饱和、状态约束等),
      并研究预设阻尼比指标在多变量耦合情形下的可解释参数化形式。
    
;观测与补偿机制的噪声鲁棒性与离散实现: 在高噪声、低采样频率条件下,
      进一步研究滑模观测器、微分跟踪器与扰动观测器的参数整定规律与稳定性裕度,
      建立面向离散实现的统一误差界评估方法,并结合执行器带宽、量化误差等因素给出系统化选参准则。
    
;抖振抑制与性能进一步提升: 探索更平滑的到达律、连续滑模或高阶滑模策略,
      以及将扰动估计与自适应机制结合,在降低抖振的同时保持鲁棒性与预设瞬态性能,
      并开展对能耗、控制输入峰值与执行器热负荷等工程指标的综合优化。
    
;面向实际应用的任务扩展与平台迁移: 将本文方法从板球平衡系统推广至更复杂的机电伺服系统、
      平台稳定系统或机器人关节伺服场景,验证在更复杂动力学、更多约束与更强耦合条件下的适用性,
      并探索与轨迹规划、学习型模型补偿等模块的融合以实现更高层次的性能与泛化能力。
    
    
    综上所述,本文围绕预设阻尼比这一可解释瞬态指标,建立了从理论判据、控制器设计到实验验证的完整链路,
    为不确定非线性系统在“稳定性 + 瞬态性能可预设 + 工程可实现”方面提供了一套可落地的方法框架。
    未来在更一般系统、更复杂约束与更强噪声环境中的进一步研究,有望推动该类方法在实际工程系统中的应用与推广。
    
    \end{conclusions}
    
    % Local Variables:
    % TeX-master: "../thesis"
    % TeX-engine: xetex
    % End:
    
