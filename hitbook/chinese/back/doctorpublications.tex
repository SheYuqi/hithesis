% !Mode:: "TeX:UTF-8" 

\begin{doctorpublication}
\noindent (一)发表的学术论文

;XXX,XXX. Static Oxidation Model of Al-Mg/C Dissipation Thermal Protection Materials[J]. Rare Metal Materials and Engineering, 2010, 39(. 1): 520-524.(~收录,号为~669,=0.16)
;XXX,XXX. 精密超声振动切削单晶铜的计算机仿真研究[J]. 系统仿真学报,2007,19(4):738-741,753.(~收录号:20071310514841)
;XXX,XXX. 局部多孔质气体静压轴向轴承静态特性的数值求解[J]. 摩擦学学报,2007(1):68-72.(~收录号:20071510544816)
;XXX,XXX. 硬脆光学晶体材料超精密切削理论研究综述[J]. 机械工程学报,2003,39(8):15-22.(~收录号:2004088028875)
;XXX,XXX. 基于遗传算法的超精密切削加工表面粗糙度预测模型的参数辨识以及切削参数优化[J]. 机械工程学报,2005,41(11):158-162.(~收录号:2006039650087)
;XXX,XXX. Discrete Sliding Mode Cintrok with Fuzzy Adaptive Reaching Law on 6-PEES Parallel Robot[C]. Intelligent System Design and Applications, Jinan, 2006: 649-652.(~收录号:20073210746529)


\noindent (二)申请及已获得的专利(无专利时此项不必列出)

;XXX,XXX. 一种温热外敷药制备方案:中国,88105607.3[P]. 1989-07-26.


\noindent (三)参与的科研项目及获奖情况

;XXX,XXX. XX~气体静压轴承技术研究, XX~省自然科学基金项目.课题编号:XXXX.
;XXX,XXX. XX~静载下预应力混凝土房屋结构设计统一理论. 黑江省科学技术二等奖, 2007.

%\vfill
%\hangafter=1\hangindent=2em\noindent
%\setlength{\parindent}{2em}
\end{doctorpublication}
